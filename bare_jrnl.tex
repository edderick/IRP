\documentclass[journal]{IEEEtran}

\usepackage{paralist}
\hyphenation{op-tical net-works semi-conduc-tor}


\begin{document}

\title{Security of the Internet of Things in the Home}
\author{Edward~Seabrook }

\maketitle


\begin{abstract}
As the number of devices, and types of devices in the modern home grows, so
does the possibility for attack. The Internet of Things generation network
technology must be designed with security and privacy considerations in mind
from the start. In this paper, a thorough review of the protocold stack,
security consideration and prevention mechanisms is given. 
\end{abstract}

\begin{IEEEkeywords}
Internet of Things, Security, IETF.
\end{IEEEkeywords}

\IEEEpeerreviewmaketitle

\section{Introduction}
\IEEEPARstart{T}{he} ``Internet of Things'' (IoT) has become somewhat of a
buzzword recently; the term is thrown around liberally, but there is no single
accepted definition of the concept. The earliest use of the term was by Kevin
Ashton in 1999, and was used to describe the use of RFID tags and the internet
in monitoring the supply chain at Procter \& Gamble. 

A few years later, in 2005, the ITU published a report titled “The Internet of
Things”, which discussed how RFID would be used to give objects from the real
world a presence on the Internet. The report added a new dimension to the
internet, suggesting that in the future connectivity would be available at
``anytime'', ``any place'' and to ``anything''.

More recent definitions of the IoT have shied away from limiting themselves to
RFID, favouring more feature rich protocols, such as 802.14.5. The one core
theme that all definitions agree on, is that as the IoT enters the mainstream,
the number of devices connected to the Internet will increase at a phenomenal
rate — Gartner estimates that by 2020, there will be more than 26 billion
devices connected to the Internet. 

In this paper, the IoT is taken to refer to the massive increase in
connectivity that we have already started to see as the mid 2010s approaches.
This conjunction of the traditional internet, and the emerging IoT, may be
referred to as ``the Internet of Everything''; it is important not to neglect
the increasing presence of more traditional devices on the Internet. 

As we enter this increasingly connected age, security becomes a serious
concern. With more devices, and more types of device, the number of attack
surfaces grows. The first generation Internet took a lax approach to security,
focusing first on achieving functionality. This has lead to a an internet that
is insecure by default; only recently have larger websites began using
encrypted sessions (through the use of TLS) to secure communications with their
users. 

There has been a large amount of attention recently placed on the IoT in a
business, or industrial, context. Discussions of wireless sensor networks (WSN)
used to monitor manufacture and agriculture, to ensure optimal production
capacity; or proposals of complex policy management frameworks, are common
place. Interesting work has also been done in the area of vehicle networks; at
Def Con 21, a horrifying demonstration showed how it is possible to take
control of a modern automobile. There are clearly important security issues in
these areas, especially with respect to critical infrastructure, however, this
paper chooses to focus on the deployment in home networks. 

Home networks offer a unique challenge — they lack funding for expensive
network equipment, typically acquiring devices from a variety of vendors; they
very rarely have access to a competent, professional systems administrator; and
they contain an increasing range of very personal devices and appliances
\cite{ACM_ModHome}. As such, the security requirements tend to be high, while
the difficulty of configuring the network to assure such a level of security
must be very low. 


\section{Technologies}
In the early days of the internet, there was a wide range of competing
standards: DECnet, TCP/IP, AppleTalk, OSI and many others.  Eventually TCP/IP
came out on top and is now universally used worldwide. Similarly, there are
many competing technology stacks for the IoT. Each have their own advantages
and disadvantages, and inspiration is distributed horizontally. 

As with the internet, it is convenient to think about the internet of things in
terms of the 7 layer OSI model\footnote{Although the TCP/IP model won the
battle, the numberings from the OSI model are still commonly used in industry}.
The TCP/IP model is built around a narrow waist of IP (either IPv4 or IPv6),
with multiple choices of protocol sitting above and below the IP layer. 

\subsection{Layer 1 -- The Physical Layer}
The physical layer defines the physical medium being used for the transmission
of messages. Traditional networks made use of wired connections for this
purpose, however, due to the expense and inconvenience, we have seen a
transition away from wired networks to wireless, radio based technologies. 

\subsubsection{IEEE 802.11}
The standard that defines WiFi, and is probably the most
familiar wireless networking protocol to home users. The latest standard for
WiFi is 802.11ac, however, its recent approval means it is yet to be widely
deployed; 802.11n is probably the most widely used revision. 

802.11 is widely used in the home as a wireless alternative to ethernet: on
laptops, mobile phones and tablets. For devices with mains connectivity, where
cost is not a major factor, and high bandwidth is required, WiFi may be a
sensible choice.  However, high power usage makes it unsuitable for the
constrained devices of the IoT. 

In terms of security, WiFi has has a shaky past. The first security protocol
for WiFi, Wired Equivalency Protocol (WEP) had some serious issues that
rendered it completely insecure. And another extension, WiFi Protected
Setup (WPS) also contained a serious vulnerability. As far as we know, the
current security standard for WiFi, WiFi Protected Access II (WPA2) is
secure. 

\subsubsection{Bluetooth}
is a wireless communications technology used in Personal Area
Networks (PANs). It operates in the Industrial, Scientific and Industrial (ISM)
band, and is used primarily for connecting mobile phones to headsets. Bluetooth
has a lower bandwidth and energy usage than WiFi, so is more suitable for
connecting constrained devices than WiFi. 

The security used in Bluetooth uses Linear Feedback Shift Registers (LFSRs) to
generate a pseudorandom keystream, so may well have serious security
weaknesses.

\subsubsection{RFID}
Much of the early work on the IoT focused on Radio-frequency Identification
(RFID). There are two basic types of RFID "tag": active tags, which are powered
by the reader; and passive tags, which have their own power source. The main
benefit of RFID is that the passive tags can be created very cheaply -- the
cheapest tags cost somewhere around 0.05USD. 

\subsubsection{802.4.15}
is probably the most interesting PHY layer technology for the IoT. It has very
low energy usage, but also a very low bandwidth. The very low energy
requirements makes 802.4.15 ideal for WSNs where the battery cannot be replaced
regularly. 

\subsection{Layer 2 - The Link Layer}
The link layer is used to determine who communicates, when. Many of the
protocols mentioned in layer 1 have link layer or MAC layer mechanisms
specified in the standards documents for the PHY layer. 

\subsubsection{ZigBee}


\subsubsection{WirelessHart}


\subsubsection{802.4.15e}



\subsection{Layer 3 - The Network Layer}
The network layer is the narrow waist of the TCP/IP stack. IPv6 is the
preferred protocol for use in to the future.

\begin{itemize}
\item IPv4 / IPv6
\item 6LoWPAN
\end{itemize}

\subsection{Layer 4 - The Transport Layer}
The transport layer runs on top of the network layer and provides multiplexing
for applications; optionally it also provides retransmission controls etc in
the form of TCP. 

A variety of factors mean that TCP is not suitible for the Internet of things,
instead we chose to use UDP, and handle the link reliability in a higher layer.

\subsection{Layer 7 - The Application Layer}
The observant reader will notice that we have skipped layers 5 and 6. This is
because they are not part of the TCP/IP model, but simply bloat in the design
by committee OSI model.


\section{Security Risks}
The internet is, for many, no longer a luxury, but a lifeline. Many of our
everyday tasks have been greatly simplified by the internet. The IoT will only
exaggerate this effect, increasing our reliance on technology. The devices that
make up the IoT that reside in our homes are likely to collect huge amounts of
sensitive information -- it i is paramount that we avoid anything happening to
this data, as the consequences could be significant. 


\section{Prevention and Mitigation}
So now we know the protocols, and the risks. Explain how the risk can be
prevented.


\section{Future Work}
What things are still open research problems. 


\section{Conclusion}
A summary and evaluation of the findings presented in the paper.



\section*{Acknowledgment}

The authors would like to thank...


% references section

% can use a bibliography generated by BibTeX as a .bbl file
% BibTeX documentation can be easily obtained at:
% http://www.ctan.org/tex-archive/biblio/bibtex/contrib/doc/
% The IEEEtran BibTeX style support page is at:
% http://www.michaelshell.org/tex/ieeetran/bibtex/
%\bibliographystyle{IEEEtran}
% argument is your BibTeX string definitions and bibliography database(s)
%\bibliography{IEEEabrv,../bib/paper}
%
% <OR> manually copy in the resultant .bbl file
% set second argument of \begin to the number of references
% (used to reserve space for the reference number labels box)
\begin{thebibliography}{1}

\bibitem{ACM_ModHome}
Denning, T., Kohno, T., \& Levy, H. M. (2013). Computer Security and the Modern Home. Communications of the ACM, 56(1), 94. 


\end{thebibliography}


%\vfill

% that's all folks
\end{document}
