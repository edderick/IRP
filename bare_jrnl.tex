\documentclass[11pt,journal,compsoc]{IEEEtran}
%\documentclass[journal]{IEEEtran}

\usepackage{paralist}
\hyphenation{op-tical net-works semi-conduc-tor}


\begin{document}

\title{Security of the Internet of Things in the Home}
\author{Edward~Seabrook }

\maketitle


\begin{abstract}
As the number of devices, and types of devices in the modern home grows, so
does the possibility for attack. The Internet of Things generation network
technology must be designed with security and privacy considerations in mind
from the start. In this paper, a thorough review of the protocold stack,
security consideration and prevention mechanisms is given. 
\end{abstract}

\begin{IEEEkeywords}
Internet of Things, Security, IETF.
\end{IEEEkeywords}

\IEEEpeerreviewmaketitle

\section{Introduction}
\IEEEPARstart{T}{he} ``Internet of Things'' (IoT) has become somewhat of a
buzzword recently; the term is thrown around liberally, but there is no single
accepted definition of the concept. The earliest use of the term was by Kevin
Ashton in 1999, and was used to describe the use of RFID tags and the internet
in monitoring the supply chain at Procter \& Gamble. 

A few years later, in 2005, the ITU published a report titled “The Internet of
Things”, which discussed how RFID would be used to give objects from the real
world a presence on the Internet. The report added a new dimension to the
internet, suggesting that in the future connectivity would be available at
``anytime'', ``any place'' and to ``anything''.

More recent definitions of the IoT have shied away from limiting themselves to
RFID, favouring more feature rich protocols, such as 802.14.5. The one core
theme that all definitions agree on, is that as the IoT enters the mainstream,
the number of devices connected to the Internet will increase at a phenomenal
rate — Gartner estimates that by 2020, there will be more than 26 billion
devices connected to the Internet. 

In this paper, the IoT is taken to refer to the massive increase in
connectivity that we have already started to see as the mid 2010s approaches.
This conjunction of the traditional internet, and the emerging IoT, may be
referred to as ``the Internet of Everything''; it is important not to neglect
the increasing presence of more traditional devices on the Internet. 

As we enter this increasingly connected age, security becomes a serious
concern. With more devices, and more types of device, the number of attack
surfaces grows. The first generation Internet took a lax approach to security,
focusing first on achieving functionality. This has lead to a an internet that
is insecure by default; only recently have larger websites began using
encrypted sessions (through the use of TLS) to secure communications with their
users. 

There has been a large amount of attention recently placed on the IoT in a
business, or industrial, context. Discussions of wireless sensor networks (WSN)
used to monitor manufacture and agriculture, to ensure optimal production
capacity; or proposals of complex policy management frameworks, are common
place. Interesting work has also been done in the area of vehicle networks; at
Def Con 21, a horrifying demonstration showed how it is possible to take
control of a modern automobile. There are clearly important security issues in
these areas, especially with respect to critical infrastructure, however, this
paper chooses to focus on the deployment in home networks. 

Home networks offer a unique challenge — they lack funding for expensive
network equipment, typically acquiring devices from a variety of vendors; they
very rarely have access to a competent, professional systems administrator; and
they contain an increasing range of very personal devices and appliances
\cite{ACM_ModHome}. As such, the security requirements tend to be high, while
the difficulty of configuring the network to assure such a level of security
must be very low. 


\section{Technologies}
In the early days of the internet, there was a wide range of competing
standards: DECnet, TCP/IP, AppleTalk, OSI and many others.  Eventually TCP/IP
came out on top and is now universally used worldwide. Similarly, there are
many competing technology stacks for the IoT. Each have their own advantages
and disadvantages, and inspiration is distributed horizontally. 

As with the internet, it is convenient to think about the internet of things in
terms of the 7 layer OSI model\footnote{Although the TCP/IP model won the
battle, the numberings from the OSI model are still commonly used in industry}.
The TCP/IP model is built around a narrow waist of IP (either IPv4 or IPv6),
with multiple choices of protocol sitting above and below the IP layer. 

\subsection{Layers 1 \& 2 -- The Physical and Data Link Layers}
The physical layer defines the physical medium being used for the transmission
of messages. Traditional networks made use of wired connections for this
purpose, however, due to the expense and inconvenience, we have seen a
transition away from wired networks to wireless, radio based technologies. The
link layer is used to determine who communicates, when.

\subsubsection{IEEE 802.11}
The standard that defines WiFi, and is probably the most
familiar wireless networking protocol to home users. The latest standard for
WiFi is 802.11ac, however, its recent approval means it is yet to be widely
deployed; 802.11n is probably the most widely used revision. 

802.11 is widely used in the home as a wireless alternative to ethernet: on
laptops, mobile phones and tablets. For devices with mains connectivity, where
cost is not a major factor, and high bandwidth is required, WiFi may be a
sensible choice.  However, high power usage makes it unsuitable for the
constrained devices of the IoT. 

In terms of security, WiFi has has a shaky past. The first security protocol
for WiFi, Wired Equivalency Protocol (WEP) had some serious issues that
rendered it completely insecure. And another extension, WiFi Protected
Setup (WPS) also contained a serious vulnerability. As far as we know, the
current security standard for WiFi, WiFi Protected Access II (WPA2) is
secure. 

\subsubsection{Bluetooth}
is a wireless communications technology used in Personal Area Networks (PANs).
It operates in the Industrial, Scientific and Industrial (ISM) band, and is
used primarily for connecting mobile phones to headsets. Bluetooth has a lower
bandwidth and energy usage than WiFi, so is more suitable for connecting
constrained devices than WiFi. Unlike WiFi, Bluetooth also defines the higher
levels for communication. 

The security used in Bluetooth uses Linear Feedback Shift Registers (LFSRs) to
generate a pseudorandom keystream, so may well have serious security
weaknesses.

\subsubsection{RFID}
Much of the early work on the IoT focused on Radio-frequency Identification
(RFID). There are two basic types of RFID "tag": active tags, which are powered
by the reader; and passive tags, which have their own power source. The main
benefit of RFID is that the passive tags can be created very cheaply -- the
cheapest tags cost somewhere around 0.05USD. 

\subsubsection{802.4.15}
is probably the most interesting PHY layer technology for the IoT. It has very
low energy usage, but also a very low bandwidth. The very low energy
requirements makes 802.4.15 ideal for WSNs where the battery cannot be replaced
regularly. 


\subsection{Layer 3 - The Network Layer}
The network layer is the narrow waist of the TCP/IP stack. On the traditional
internet IPv6 is the preferred protocol for use in to the future.

\subsubsection{IP}
The Internet Protocol (IP) has two major versions: IPv4 and IPv6. Although most
networks at present still make use of IPv4, we have now run out of IPv4
addresses, so in the future we will be forced to use IPv6.

IP is an important protocol as it is used almost universally by machines
connected to the Internet. An IP network does not need to know about the higher
level protocols to carry IP traffic, an IP packet is also no concerned about
the lower levels that are carrying it. 

\subsubsection{6LoWPAN}
Unfortunately IP as a protocol was not designed with highly constrained
embedded devices in mind, as a result, it can be too resource intensive to run
on many "things". 6LoWPAN is a protocol that is designed to work around this
problem, by providing a bare bones version of IPv6. It includes features such
as header compression and packet fragmentation.  

\subsubsection{ZigBee}
ZigBee is a set of protocols that runs over 802.15.4. It is designed to be a
low cost option for wireless networks that require low power usage, and have no
bandwidth requirements. ZigBee defines the network layer, the application layer
and the high level concept of a ZigBee device object (ZDO). 

\subsubsection{WirelessHART}
WirelessHART is also built on top of 802.14.5, it is designed to be used for
industrial automation. 

\subsection{Layer 4 - The Transport Layer}
The transport layer runs on top of the network layer and provides multiplexing
for applications; optionally it also provides retransmission controls in
the form of TCP. 

A variety of factors mean that TCP is not suitable for the low power lossy
networks (LLNs) of the IoT, instead we choose to use UDP, handling link
reliability in a higher layer.

\subsection{Layer 7 - The Application Layer}
The application layer is probably the widest layer, since any application can
define it's own application layer if it wishes. There are many familiar
application layer protocols that we use on a daily basis, there include SSH,
FTP, SMTP, and DNS. 

\subsubsection{HTTP} 
With the advent of the World Wide Web (W3), it has been observed that HTTP has
become the de facto standard for all applications. The concept of a RESTful
webservice is widely applied.  

\subsubsection{CoAP}
HTTP was never designed to be a light weight protocol; instead the focus on
human readability is one of the things that is likely to have contributed to
the massive success it has had. For constrained devices, HTTP is simply too
bandwidth, and thus power, intensive. The Constrained Applications Protocol
(CoAP) is designed as a compatible subset of HTTP. It is not a naive
compression of HTTP, but rather a completely new protocol with a stateless
one-to-one mapping to (and from) HTTP. 

\subsection{Overall Architecture}
At present it seems likely that an IP based Internet of Things will become a
reality very soon. If this is the case, then the architecture detailed in
\cite{Palattella2013} is a good candidate as a standards protocol stack. This
architecture uses 802.15.4 as the PHY and MAC layers; 6LoWPAN as the network
layer; UDP for the transport layer; and CoAP as the Application layer. This
architecture is a good choice because it offers excellent compatibility, and
thus connectivity, to the traditional Inernet, while avoiding impacting the
performance of the IoT with heavy, power intensive protocols. 

\section{Security Goals}
The internet is, for many, no longer a luxury, but a lifeline. Many of our
everyday tasks have been greatly simplified by the internet. The IoT will only
exaggerate this effect, increasing our reliance on technology. The devices that
make up the IoT that reside in our homes are likely to collect huge amounts of
sensitive information -- it is paramount that we avoid anything happening to
this data, as the consequences could be significant. 

As we transition from an Internet containing only virtual entities to an
Internet consisting of real world objects as well, the damages that will be
possible through attack or data leak will increase. There are a number of
security and privacy goals that must be met for users to be able to place trust
in the Internet of Things.

\subsection{Confidentiality}
A confidential message should only be readable by its intended recipients.
Mechanisms are needed to ensure that attackers are not able to intercept and
understand any messages that are transmitted. The level of confidentiality
required varies depending on the use case: for example, a temperature sensor in
a public place probably doesn't convey any information that an attacker
couldn't collect for themselves; whereas a medical sensor for a potentially
embarrassing medical condition will probably have quite high confidentiality
requirements. Wireless communication is particularly susceptible to
interception, this should not reduce the confidentiality of the messages.  

\subsection{Integrity}
Integrity is the concept that a message should be exactly the same when it is
received as it was when it was sent. Mechanisms are required to ensure that a
message does not change during transmission, if the message does change in
transit, it should be either automatically corrected, or discarded. A message
could be altered accidentally, through some error in the transmission process,
or it could be altered maliciously by an attacker. As the IoT will likely rely
mainly on wireless communications, both malicious and accidental changes are likely. 

\subsection{Availability}
Availability refers to the ability to make use of a service when it is needed.
On the regular Internet we are familiar with Denial of Service (DoS) attacks,
which render websites or networks completely unusable. The same kind of attacks
would be possible both against, and from the thing of the internet of things.
In early 2014, Proofpoint found that many low power devices, including one
smart refrigerator were being used to send spam \cite{Proofpoint2014}, a
Distributed DoS attack could be launched in a similar fashion.  

\subsection{Non-repudiation}
Non-repudiation mechanisms ensure that if an action was performed by somebody,
then it is impossible for them to claim that they had not performed the action.
(Right now I can't think of any reason this would actually be needed in the
IoT) 

\subsection{Authenticity}
Authenticity is assuring that a message has been sent by the party that it
claims to have been sent by. In the IoT protocols should provide a way of
preventing attackers, for example, from pretending to be sensor nodes, and
submitting bogus readings, that could lead to undesired actions.  

\subsection{Authorization}
Authorization is ensuring that the party sending a command is actually allowed
to make the request that they have made. Presently, we often see WiFi routers
and other embedded devices that ship with well known standard default usernames
and password. This makes it very easy for an attacker to guess the login
credentials. If things do not change, then we are likely to see the same
phenomenon in the IoT. 


\section{Security Threats}
It is difficult to develop a strategy of prevention and mitigation without
first understanding some of the potential attacks that should be defended
against. I may merge this section back into the one above, it makes a lot of sense to... 

There is a good list here \cite{Ning2013}


\section{Prevention and Mitigation}
So now we know the protocols, and the risks. Explain how the risk can be
prevented.

\begin{itemize}
\item DTLS
\item per hop encryption algorithms
\end{itemize}


\section{Future Work}
What things are still open research problems. 

\begin{itemize}
\item How do we force vendors to ship devices with secure password?
\item How do we educate users?
\end{itemize}


\section{Conclusion}
A summary and evaluation of the findings presented in the paper.



\section*{Acknowledgment}

The authors would like to thank...


% references section

% can use a bibliography generated by BibTeX as a .bbl file
% BibTeX documentation can be easily obtained at:
% http://www.ctan.org/tex-archive/biblio/bibtex/contrib/doc/
% The IEEEtran BibTeX style support page is at:
% http://www.michaelshell.org/tex/ieeetran/bibtex/
%\bibliographystyle{IEEEtran}
% argument is your BibTeX string definitions and bibliography database(s)
%\bibliography{IEEEabrv,../bib/paper}
%
% <OR> manually copy in the resultant .bbl file
% set second argument of \begin to the number of references
% (used to reserve space for the reference number labels box)
\begin{thebibliography}{1}

\bibitem{ACM_ModHome}
Denning, T., Kohno, T., \& Levy, H. M. (2013). Computer Security and the Modern Home. Communications of the ACM, 56(1), 94. 


\bibitem{Palattella2013}
Palattella, M. R., Accettura, N., Vilajosana, X., Watteyne, T., Grieco, L. A., Boggia, G., \& Dohler, M. (2013). Standardized Protocol Stack for the Internet of (Important) Things. IEEE Communications Surveys \& Tutorials, 15(3), 1389–1406.ng2013 

\bibitem{Ning2013}
Ning, H., Liu, H., \& Yang, L. T. (2013). Cyberentity Security in the Internet of Things. Computer, 46(4), 46–53. 

\bibitem{Proofpoint2014}
http://www.proofpoint.com/about-us/press-releases/01162014.php


\end{thebibliography}


%\vfill

% that's all folks
\end{document}
